\documentclass[12pt]{article}
\usepackage[T2A]{fontenc}
\usepackage[utf8]{inputenc} 
\usepackage[russian, english]{babel}
\usepackage{amsmath,}
\usepackage{booktabs}
\usepackage{graphicx}
\usepackage[table, dvipsnames]{xcolor}
\graphicspath{ {} }
\author{Моргачев Глеб 577}
\title{\textbf{Лабораторная работа \textnumero 1.2} \\{Исследование эффекта Комптона}}
\date{}


\begin{document}
	\maketitle

	С помощью сцинтиляционного спектографа исследуется энергетический спектр $\gamma$\--квантов.
	Определяется энергии рассенных $\gamma$\--квантов в зависимости от угла рассеяния, а также энергия покоя частицы, на которой происходит комптоновское рассеяние.
	\section*{Теория}
	
		Эффект Комптона \-- увеличение длины рассеянного излучения по сравнению с падающим.
		
		Будем считать, что $\gamma$\--излучение \-- представляет собой поток квантов у которых:
		\begin{eqnarray}
		    E = \hbar\omega \\
			p = \dfrac{\hbar\omega}{c}
		\end{eqnarray}
        При этом эффект Комптона интерпретируется как результат упругого соударения двух частиц: $\gamma$\--кванта и свободного электрона. 
		Пусть электрон до соударние покоился, его энергия 
		\begin{equation}
			E_{el} = mc^2
		\end{equation}
		$\gamma$\--квант имел начальную энергию 
		\begin{eqnarray}
		    E_k = \hbar\omega_0 \\
			p_k = \frac{\hbar\omega}{c}
		\end{eqnarray}
		Тогда после соударения:
		\begin{eqnarray}
			E_{el} = \gamma mc^2 \\
			p_{el} = \gamma mv
		\end{eqnarray}
		Здесь $\gamma = \left(1 - \left(\dfrac{v}{c}\right)^2\right)^\frac{1}{2}$.
		$\gamma$\--квант рассеялся на угол $\theta$ к первоначальному направлению движения.
		$\phi$ \-- угол, под которым полетел электрон после соударения.
		
		Тогда ЗСИ и ЗСЭ:
		\begin{eqnarray}
			mc^2 + \hbar\omega_0 = \gamma mc^2 + \hbar\omega_1 \\
			\dfrac{\hbar\omega_0}{c} = \gamma mv\cos(\phi) + \dfrac{\hbar\omega_1}{c}\cos(\theta) \\
			\gamma mv \sin(\phi) = \dfrac{\hbar\omega}{c}\sin{\theta}
		\end{eqnarray}
		Отсюда:
		\begin{equation}
			\Delta\lambda = \dfrac{h}{mc}(1 - \cos(\theta)) = \Lambda_k(1 - \cos(\theta))
		\end{equation}
		Здесь $\Lambda_k$ \-- Комптоновская длинна волны электрона.
		Последнее равенстно можно переписать в виде:
		\begin{equation}
			\dfrac{1}{\epsilon(\theta)} - \dfrac{1}{\epsilon_0} = 1 - \cos(\theta)
		\end{equation}
		Здесь $\epsilon_0 = E_0/(mc^2)$ \-- энергия $\gamma$\--кванта, падающего на рассеиватель.
		
		\section*{Измерения}
		В ходе эксперимента различным уровням энергии будут соответствовать различные каналы $N$, соответствующие вершинам фотопиков.
		Таким образом:
		\begin{equation}
			\dfrac{1}{N(\theta)} - \dfrac{1}{N_0} = A(1 - \cos(\theta))
		\end{equation}
		Оценим погрешности:
		\begin{eqnarray}
			\dfrac{\sigma N}{N} = 0.01 \\
			\sigma \theta = 0.5 \deg
			\\
			\sigma X = |\sin(\theta)|\sigma\theta \\
			\sigma Y = \left|\dfrac{\sigma N}{N(\theta)^2}\right|
		\end{eqnarray}
	\begin {table}
 	\caption {Результаты измерений} \label{tab:title} 
	\begin{center}
	
{\rowcolors{2}{SkyBlue!30!OliveGreen!60}{OliveGreen!60!SkyBlue!10}
		\begin{tabular}{lrrrrrr}
	\toprule
	
&	deg & N & X & Y & $\sigma(X)$ & $ \sigma(Y)$ \\
\midrule
0  &   0.0 &  901 &  0.0000 &  0.001110 &     0.0000 &     0.000006 \\
1  &   5.0 &  945 &  0.0038 &  0.001058 &     0.0008 &     0.000006 \\
2  &  10.0 &  904 &  0.0151 &  0.001106 &     0.0015 &     0.000006 \\
3  &  20.0 &  840 &  0.0603 &  0.001190 &     0.0030 &     0.000007 \\
4  &  30.0 &  797 &  0.1339 &  0.001255 &     0.0043 &     0.000008 \\
5  &  40.0 &  729 &  0.2339 &  0.001372 &     0.0056 &     0.000009 \\
6  &  50.0 &  639 &  0.3572 &  0.001565 &     0.0066 &     0.000012 \\
7  &  60.0 &  562 &  0.5000 &  0.001779 &     0.0075 &     0.000016 \\
8  &  70.0 &  474 &  0.6579 &  0.002110 &     0.0082 &     0.000022 \\
9  &  80.0 &  435 &  0.8263 &  0.002299 &     0.0085 &     0.000026 \\
10 &  90.0 &  401 &  1.0000 &  0.002494 &     0.0087 &     0.000031 \\
11 &  -5.0 &  906 &  0.0038 &  0.001104 &     0.0007 &     0.000006 \\
12 &   5.0 &  959 &  0.0038 &  0.001043 &     0.0007 &     0.000005 \\
\bottomrule
		\end{tabular}}
	
	\end{center}
\end{table}
\\\bigskip
\section*{Результаты}
	
\noindent\makebox[\textwidth]{\includegraphics[width = 1.3\textwidth]{graph_1}}
Параметры получившейся наилучшей прямой (использованный метод апроксимации \-- OLS (ordinary least square)):
	\begin{eqnarray}
		Y = AX + B \\
		A = 0.0015 \pm 3*10^{-5} \\
		B = 0.0011 \pm 1*10^{-5}
	\end{eqnarray}
	
	Также, нужно учесть погрещность определения угла и максимума,
	посколько статистическая погрешность мала по сравнению с упомянутой погрешностью определения максимумов, то оценим:
	\begin{eqnarray}
		\sigma(N_{best}) \leq 9
	\end{eqnarray}
	Отсюда, 
	%FIX
	\begin{eqnarray}
		N_{best}(0) = 931 \pm 9 \\
		N_{best}(90) = 395 \pm 9 \\
		mc^2 = E_\gamma * \dfrac{N_{best}(90)}{N_{best}(0) - N_{best}(90)} \\
		E_\gamma = 661.7 \pm keV \\
	\end{eqnarray}
	Получим:
	\begin{equation}
		mc^2 = 487 \pm 35 keV
	\end{equation}
	
\section*{Вывод}
	Приведённый способ, позволяет определить энергию покоя электрона с хорошей точностью.
	Табличное значение энергии покоя:
	\begin{equation}
		E_t = 508.5 keV
	\end{equation}
	
	С учётом погрешности, полученное нами значение энергии покоя электрона совпадает с табличным.
	

\end{document}
